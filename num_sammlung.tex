\documentclass[11pt]{scrartcl}
\usepackage[utf8]{inputenc}
\usepackage{mathtools}
  
\DeclarePairedDelimiter\abs{\lvert}{\rvert}%
	
\begin{document}

%%%%%%%%%%%%%%%%%%%%%%%%%%%%%%%%%%%%%%%%%%%%%%%%%%%%%%%%%%%%%%%%%%%%%%%%%%%%%%%%%%%%%%%%%%%% 
%%%%%%%%%%%%%%%%%%%%%%%%%%%%%%%%%%%%%%%%%%%%%%%%%%%%%%%%%%%%%%%%%%%%%%%%%%%%%%%%%%%%%%%%%%%%
\section{Grundlagen}

\pagebreak

%%%%%%%%%%%%%%%%%%%%%%%%%%%%%%%%%%%%%%%%%%%%%%%%%%%%%%%%%%%%%%%%%%%%%%%%%%%%%%%%%%%%%%%%%%%% 
%%%%%%%%%%%%%%%%%%%%%%%%%%%%%%%%%%%%%%%%%%%%%%%%%%%%%%%%%%%%%%%%%%%%%%%%%%%%%%%%%%%%%%%%%%%%
\section{Eigenwerte}

A muss regulär sein

%-----------------------------------------------
\subsection{Von-Mises Iteration}

$x$ ist zufällig gewählt, dann geht $x^k$ gegen ein Vielfaches eines Eigenvektors u von A.
$z_i^{k+1}$ nähert sich dem größten Eigenwert für $k\rightarrow \infty$.

\begin{equation}
z^k:=Ax^{k-1}
\end{equation}

\begin{equation}
x^k:=\frac{z^k}{z_{i_k}}
\end{equation}
, wobei gilt:

\begin{equation}
\abs{{z_{i_k}}^i}=\max_{i=1}^m\abs z_i^k
\end{equation}

%-----------------------------------------------
\subsection{Inverse Iteration von Wielandt}

$y$ ist zufällig gewählt, dann geht $x^k$ gegen ein Vielfaches eines Eigenvektors u von A.
$w_i^{k+1}$ nähert sich $1/\lambda_m$ für $k\rightarrow \infty$.

\begin{equation}
w^k:=A^{-1}y{k-1}
\end{equation}

\begin{equation}
y^k:=\frac{w^k}{w_{i_k}^k}
\end{equation},
wobei gilt:

\begin{equation}
\abs{{w_{i_k}}^i}=\max_{i=1}^m\abs w_i^k
\end{equation}

$A^{-1}$ wird typischerweise nicht berechnet, sondern $Aw^k=y^{k-1}$ gelöst.

\pagebreak

%%%%%%%%%%%%%%%%%%%%%%%%%%%%%%%%%%%%%%%%%%%%%%%%%%%%%%%%%%%%%%%%%%%%%%%%%%%%%%%%%%%%%%%%%%%% 
%%%%%%%%%%%%%%%%%%%%%%%%%%%%%%%%%%%%%%%%%%%%%%%%%%%%%%%%%%%%%%%%%%%%%%%%%%%%%%%%%%%%%%%%%%%%
\section{Newton-Verfahren}

\begin{equation}
x^{k+1}=x^k-\frac{F(x^k)}{F'(x^k)}
\end{equation}

Konvergiert im 1D Fall, wenn F in der Nähe von $x*$ zweifach stetig diff'bar ist.
Oder mehrdimensionale Schreibweise:

\begin{equation}
x^{k+1}=x^k-F'(x^k)^{-1}{F(x^k)}
\end{equation}

\pagebreak


%%%%%%%%%%%%%%%%%%%%%%%%%%%%%%%%%%%%%%%%%%%%%%%%%%%%%%%%%%%%%%%%%%%%%%%%%%%%%%%%%%%%%%%%%%%% 
%%%%%%%%%%%%%%%%%%%%%%%%%%%%%%%%%%%%%%%%%%%%%%%%%%%%%%%%%%%%%%%%%%%%%%%%%%%%%%%%%%%%%%%%%%%%
\section{Quadratur}

\subsection{Trapezregel}

Einfache Regel:
\begin{equation}
{\int^b_a}{f(x) dx}=(b-a)\frac{f(a)+f(b)}{2}+R(f)
\end{equation}

Zusammengesetzte Regel:
\begin{equation}
{\int^b_a}{f(x) dx}=\frac{h}{2} \cdot f(a) + h \cdot {\sum_{i=1}^{n-1}} \{ f(x_i) \} + \frac{h}{2} \cdot f(b) + R_h(f)
\end{equation}
Bedingung: $h:=\frac{b-a}{n}$, sowie $x_i:=a+ih$

\subsection {Rechteckregel}
\begin{equation}
{\int^b_a}{f(x) dx}=h \cdot {\sum_{i=0}^{n-1}} f(x_i+\frac{h}{2}) + R_h(f)
\end{equation}
Bedingung: $h:=\frac{b-a}{n}$, sowie $x_i:=a+ih$

 
\subsection {Newton-Cotes-Formel}
\begin{equation}
{\int^b_a}{f(x) dx}=(b-a)\cdot {\sum_{i=0}^n}{a_i^{(n)}f(x_i)} + R_n(f)
\end{equation}

\begin{equation}
a_i^{(n)}:=\frac{1}{b-a} {\int_{a}^b} {l_i^{(n)}(x) dx} = \frac{h}{b-a} {\int_0^n} {\prod_{\mu=0, \mu!=i}^n} \frac{s-\mu}{i-\mu}ds
\end{equation}

\begin{table}[h!]
	\centering
	\caption{Koeffizienten}
	\begin{tabular}{ccccc}
		n & $a_0^{(n)}$ & $a_1^{(n)}$ & $a_2^{(n)}$ & $a_3^{(n)}$ &
		\hline 
		1 & $\frac{1}{2}$ & $\frac{1}{2}$ & &  &
		2 & $\frac{1}{6}$ & $\frac{4}{6}$ & $\frac{1}{6}$ & &
		3 & $\frac{1}{8}$ & $\frac{3}{8}$ & $\frac{3}{8}$ & $\frac{1}{8}$
	\end{tabular}
\end{table}

Beispiel für n=2:

\begin{equation}
{\int^b_a}{f(x) dx}=\frac{(b-a)}{6} ( f(a) + 4f(\frac{a+b}{2}) + f(b) ) + R_2(f)
\end{equation}

\end{document}